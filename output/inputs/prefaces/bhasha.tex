We are extremely happy to bring forth the \textbf{1st Workshop on Benchmarks, Harmonization, Annotation, and Standardization for Human-Centric AI in Indian Languages (BHASHA 2025)} as part of \textit{IJCNLP-AACL 2025} conference held during 20th-24th December, 2025. \\

India, despite being a linguistically rich country with 22 official languages, does not enjoy the
benefits of NLP research according to the potential. The special nature of Indian languages,
from being inflectional and agglutinative to having a free word order, does not let direct usage
of tools built for other languages. In this context, the BHASHA workshop is conceived
to focus on creating
tools, benchmarks, resources, annotated corpora, evaluation metrics, etc. for Indian languages. \\

BHASHA is being held as a full-day workshop on \emph{23rd December, 2025.}
The program includes two invited talks, multiple research paper oral and poster presentations. In addition, two shared task competitions were held as part of the BHASHA workshop and papers for those will be presented as posters and demonstrations along with a shared task overview talk. \\

The program committee consisted of 19 eminent researchers from both academia and industry.
A total of 26 papers were submitted, out of which 1 was desk rejected. Of the remaining 25 papers, 11 have been accepted to be part of the proceedings, giving an overall acceptance ratio of 11/26 = 42\%.
While 8 of these papers are being presented orally, two poster sessions are held where all the 11 posters are presented for longer and better interactions between the authors and the audience.
Out of the 11 accepted papers, 8 are from India, while 1 each are from Japan, Canada, and USA. \\

The BHASHA workshop also featured two \emph{shared tasks}, one on \textbf{Grammar Error Correction (IndicGEC)} on 5 Indian languages---Hindi, Bangla, Telugu, Tamil, and Malayalam---and the other on \textbf{Word Grouping (IndicWG)} on Hindi.
While 14 and 2 teams participated respectively in the two tasks for the final stages, 10 papers were received. Out of these, 6 were accepted for the proceedings.
A summary paper on the two shared tasks and the different submissions is also included in the proceedings. \\

We thank the IJCNLP-AACL workshop chairs for helping us in various stages of the workshop.
It is my pleasure to also thank the entire organizing team and the different chairs who played their roles to perfection for the successful conduct of this workshop. \\\\

\textbf{Arnab Bhattacharya} \\
\textit{General Chair, BHASHA 2025}

